\documentclass[a4paper]{article}
%\usepackage[utf8]{vietnam}
\usepackage[utf8]{inputenc}
\usepackage{anyfontsize,fontsize}
\changefontsize[13pt]{13pt}
\usepackage{commath}
\usepackage[d]{esvect}
\usepackage{parskip}
\usepackage{xcolor}
\usepackage{amssymb}
\usepackage{slashed,cancel}
\usepackage{indentfirst}
\usepackage{pdfpages}
\usepackage{graphicx}
\usepackage{upgreek}
\usepackage{nccmath,nicematrix}
\usepackage{mathtools}
\usepackage{amsmath,systeme,amsthm,amsfonts}
\usepackage[thinc]{esdiff}
\usepackage{hyperref}
\usepackage{bm,physics}
\usepackage{fancyhdr}


\pagestyle{fancy}
\fancyhf{}
\renewcommand{\headrulewidth}{0pt}
\fancyhfoffset{0pt}
\fancyfoot[L]{Theoretical Physics}
\fancyfoot[R]{\thepage}



\usepackage[most,many,breakable]{tcolorbox}
\definecolor{mytheorembg}{HTML}{F2F2F9}
\definecolor{mytheoremfr}{HTML}{00007B}

\usepackage{geometry}
\geometry{
  a4paper,
  tmargin=2cm,rmargin=1in,lmargin=1in,margin=0.85in,bmargin=2cm,footskip=.2in
}

\definecolor{doc}{RGB}{0,60,110}
\definecolor{myg}{RGB}{56, 140, 70}
\definecolor{myb}{RGB}{45, 111, 177}
\definecolor{myr}{RGB}{199, 68, 64}
\definecolor{mytheorembg}{HTML}{F2F2F9}
\definecolor{mytheoremfr}{HTML}{00007B}
\definecolor{mylemmabg}{HTML}{FFFAF8}
\definecolor{mylemmafr}{HTML}{983b0f}
\definecolor{mypropbg}{HTML}{f2fbfc}
\definecolor{mypropfr}{HTML}{191971}
\definecolor{myexamplebg}{HTML}{F2FBF8}
\definecolor{myexamplefr}{HTML}{88D6D1}
\definecolor{myexampleti}{HTML}{2A7F7F}
\definecolor{mydefinitbg}{HTML}{E5E5FF}
\definecolor{mydefinitfr}{HTML}{3F3FA3}
\definecolor{notesgreen}{RGB}{0,162,0}
\definecolor{myp}{RGB}{197, 92, 212}
\definecolor{mygr}{HTML}{2C3338}
\definecolor{myred}{RGB}{127,0,0}
\definecolor{myyellow}{RGB}{169,121,69}
\definecolor{myexercisebg}{HTML}{F2FBF8}
\definecolor{myexercisefg}{HTML}{88D6D1}

\newcommand{\image}[1]{
\begin{figure}[H]
  \centering
  \includegraphics[width=8.0cm,height=5.0cm]{pic/#1}
  \label{#1}
\end{figure}
}
\renewcommand{\l}{\ell}
\newcommand{\dps}{\displaystyle}
\newcommand{\mean}[1]{\langle{#1}\rangle}
\newcommand{\f}[2]{\dfrac{#1}{#2}}
\newcommand{\at}[2]{\bigg\rvert_{#1}^{#2} }

\newcommand{\Qed}{\begin{flushright} \end{flushright}}

\newcommand{\solve}[1]{\setlength{\parindent}{0cm}\textbf{\textit{Solution: }}\setlength{\parindent}{1cm}#1 \Qed}

\renewcommand{\baselinestretch}{2.0}

\tcbuselibrary{theorems,skins,hooks}
\newtcbtheorem{problem}{Problem}
{%
	enhanced,
	breakable,
	colback = mytheorembg,
	frame hidden,
	boxrule = 0sp,
	borderline west = {2pt}{0pt}{mytheoremfr},
	sharp corners,
	detach title,
	% before upper = \tcbtitle\par\smallskip,
  before upper* = \tcbtitle\par\smallskip,
	coltitle = mytheoremfr,
	fonttitle = \bfseries\sffamily,
	description font = \mdseries,
	separator sign none,
	segmentation style={solid, mytheoremfr},
}
{p}

\newcommand{\mprob}[3]{%
  \begin{problem}{#1}{#2}
    #3
  \end{problem}
}

\tcbuselibrary{theorems,skins,hooks}
\newtcbtheorem[number within=section]{theorem}{Theorem}
{%
	enhanced,
	breakable,
	colback = mytheorembg,
	frame hidden,
	boxrule = 0sp,
	borderline west = {2pt}{0pt}{mytheoremfr},
	sharp corners,
	detach title,
	before upper = \tcbtitle\par\smallskip,
	coltitle = mytheoremfr,
	fonttitle = \bfseries\sffamily,
	description font = \mdseries,
	separator sign none,
	segmentation style={solid, mytheoremfr},
}
{th}

%================================
% DEFINITION BOX
%================================
\newtcbtheorem{definition}{Definition}{%
  enhanced,
  before skip=2mm,
  after skip=2mm,
  colback=red!5,
  colframe=red!80!black,
  boxrule=0.5mm,
  attach boxed title to top left={xshift=1cm,yshift*=1mm-\tcboxedtitleheight},
  boxed title style={
    colback=red!75!black,
    frame code={
      \path[fill=tcbcolback]
        ([yshift=-1mm,xshift=-1mm]frame.north west)
        arc[start angle=0,end angle=180,radius=1mm]
        ([yshift=-1mm,xshift=1mm]frame.north east)
        arc[start angle=180,end angle=0,radius=1mm];
      \path[left color=tcbcolback!60!black,
            right color=tcbcolback!60!black,
            middle color=tcbcolback!80!black]
        ([xshift=-2mm]frame.north west) --
        ([xshift=2mm]frame.north east)
        [rounded corners=1mm] --
        ([xshift=1mm,yshift=-1mm]frame.north east) --
        (frame.south east) --
        (frame.south west) --
        ([xshift=-1mm,yshift=-1mm]frame.north west)
        -- cycle;
    },
  },
  fonttitle=\bfseries,
}{def}
\newcommand{\dfn}[2]{%
  \begin{definition}{#1}{}
  #2
  \end{definition}
}

%================================
% Proof
%================================
\newenvironment{myproof}[1][\proofname]{%
  \proof[\bfseries Proof: #1]%
  \par\nobreak\smallskip
}{\endproof}

\newcommand{\pf}[2]{%
  \begin{myproof}[#1]
  #2
  \end{myproof}
}

%================================
% NOTE BOX
%================================

\usetikzlibrary{arrows,calc,shadows.blur}
\tcbuselibrary{skins}
\newtcolorbox{note}[1][]{%
	enhanced jigsaw,
	colback=gray!20!white,%
	colframe=gray!80!black,
	size=small,
	boxrule=1pt,
	title=\textbf{Note:},
	halign title=flush center,
	coltitle=black,
	breakable,
	drop shadow=black!50!white,
	attach boxed title to top left={xshift=1cm,yshift=-\tcboxedtitleheight/2,yshifttext=-\tcboxedtitleheight/2},
	minipage boxed title=1.5cm,
	boxed title style={%
		colback=white,
		size=fbox,
		boxrule=1pt,
		boxsep=2pt,
		underlay={%
		\coordinate (dotA) at ($(interior.west) + (-0.5pt,0)$);
		\coordinate (dotB) at ($(interior.east) + (0.5pt,0)$);
		\begin{scope}
						\clip (interior.north west) rectangle ([xshift=3ex]interior.east);
						\filldraw [white, blur shadow={shadow opacity=60, shadow yshift=-.75ex}, rounded corners=2pt] (interior.north west) rectangle (interior.south east);
					\end{scope}
					\begin{scope}[gray!80!black]
						\fill (dotA) circle (2pt);
						\fill (dotB) circle (2pt);
					\end{scope}
				},
		},
	#1,
}
\newcommand{\nt}[1]{\begin{note}#1\end{note}}

\title{\Huge{A solution on Mathematic Methods}}

\hypersetup{
  colorlinks=true,
	linkcolor=blue,
	filecolor=magenta,
	urlcolor=blue!70!red,
  pdftitle={MM},
  pdfpagemode=FullScreen,
}

\urlstyle{same}

\begin{document}
\setlength{\parindent}{20pt}
\newpage
\author{Tran Khoi Nguyen \\ Theoretical Physics}
\maketitle
\thispagestyle{fancy}
% \tableofcontents

\section*{Problem sheet 1}

\begin{problem}{David Skinner, 1B Methods, Page 1}{p1}
\begin{itemize}
	\item Given $f(\theta) = (\theta^{2} - \pi^{2})^{2} $, where $\theta \in [-\pi,\pi)$.
	\item Given $f(\theta) = e^{\theta}$, where $\theta \in [-\pi,\pi)$.
	\item Given $f(\theta) = \theta e^{i\theta}$, where $\theta \in [-\pi,\pi)$.
\end{itemize}
Find the Fourier series.
\end{problem}
\solve{
(a)\quad$f(\theta)$ is even function. Therefore, $b_{n} = 0$, the Fourier coefficient $a_{n}$ can be written as
\begin{equation}
	\begin{aligned}
		a_{n}
		 & = \frac{1}{\pi} \int_{-\pi}^{\pi} \cos(n \theta)  f(\theta) d\theta                   \\
		 & = \frac{1}{\pi} \int_{-\pi}^{\pi} \cos(n \theta)  (\theta^{2} - \pi^{2})^{2} d\theta.
	\end{aligned}
\end{equation}
Let
\begin{gather}
	\begin{cases}
		u =  (\theta^{2} - \pi^{2})^{2} \\
		dv = \cos \left(n \theta\right) d\theta
	\end{cases}
	\Rightarrow
	\begin{cases}
		du = \left(4 \theta^{3} - 4 \pi^{2} \theta\right) d\theta \\
		v = \dfrac{\sin(n \theta)}{n}
	\end{cases}
\end{gather}
we get
\begin{equation}
	\begin{aligned}
		a_{n}
		 & = \cancelto{0}{ \evaluated{\frac{\left( \theta^{ 2 } - \pi^{ 2 }  \right)^{ 2 }}{2\pi}  \frac{\sin n \theta}{n}}_{ -\pi }^{ \pi }} - \frac{1}{2\pi}\int_{-\pi}^{\pi} \frac{\sin n \theta}{n} \left(4 \theta^{3} - 4 \pi^{2} \theta \right) d\theta,
	\end{aligned}
\end{equation}
let
\begin{gather}
	\begin{cases}
		u =  (4 \theta^{3} - 4\pi^{2}\theta) \\
		dv = -\dfrac{\sin \left(n \theta\right)}{n} d\theta
	\end{cases}
	\Rightarrow
	\begin{cases}
		du = \left( 12 \theta^{2} - 4 \pi^{2} \right) d\theta \\
		v = \dfrac{\cos n \theta}{n^{2}}
	\end{cases}
\end{gather}
we get
\begin{equation}
	\begin{aligned}
		a_{n} & = \cancelto{0}{\evaluated{( 4 \theta^{3} - 4 \pi^{2} \theta ) \frac{\cos n \theta}{n^{2}}}_{-\pi}^{\pi}} - \frac{1}{2\pi}\int_{-\pi}^{\pi} \frac{\cos n \theta}{n^{2}}  \left( 12 \theta^{2} - 4 \pi^{2} \right) d\theta, \\
	\end{aligned}
\end{equation}
let
\begin{gather}
	\begin{cases}
		u = \left( 12 \theta^{2} - 4 \pi^{2} \right) \\
		dv = - \dfrac{\cos n \theta}{n^{2}} d\theta
	\end{cases}
	\Rightarrow
	\begin{cases}
		du = 24 \theta d\theta \\
		v = -\dfrac{\sin n \theta}{n^{3}}
	\end{cases}
\end{gather}
we get
\begin{equation}
	\begin{aligned}
		a_{n} & = \evaluated{ -( 12 \theta^{2} - 4 \pi^{2} ) \frac{\sin n \theta}{n^{3}}}_{-\pi}^{\pi} + \int_{-\pi}^{\pi} \frac{\sin n \theta}{n^{3}} 24 \theta d\theta,
	\end{aligned}
\end{equation}
let
\begin{gather}
	\begin{cases}
		u =  24 \theta \\
		dv = \dfrac{\sin n \theta}{n^{3}}
	\end{cases}
	\Rightarrow
	\begin{cases}
		du = 24 d\theta \\
		v = -\dfrac{\cos n \theta}{n^{4}}
	\end{cases}
\end{gather}
the $a_{n}$ can be written as
\begin{equation}
	\begin{aligned}
		a_{n}
		 & = \evaluated{- \dfrac{24}{\pi} \theta \dfrac{\cos n \theta}{n^{4}}}_{-\pi}^{\pi} + \dfrac{24}{\pi} \int_{-\pi}^{\pi} \dfrac{\cos n \theta}{n^{4}} \\
		 & = -\dfrac{24}{n^{4}\pi} ( \pi \cos n \pi - ( - \pi \cos( -n\pi) ) )                                                                               \\
		 & = - \frac{48}{n^{4}} ( -1 )^{n},
	\end{aligned}
\end{equation}
and $a_{0}$ is
\begin{equation}
	\begin{aligned}
		a_{0} & = \frac{1}{\pi} \int_{-\pi}^{\pi} f ( \theta ) d\theta             \\
		      & = \frac{2}{\pi} \int_{0}^{\pi} ( \theta^{2}  - \pi^{2})^{2} d\theta \\
		      & = \frac{16 \pi^{4}}{15}.
	\end{aligned}
\end{equation}
The Fourier series for $f$ can be written as
\begin{equation}
	\begin{aligned}
		f(\theta) & = \frac{8 \pi^{4}}{15} - \sum_{n = 1}^{\infty} \frac{48}{n^{4}} ( -1 )^{n} \cos n\theta.
	\end{aligned}
\end{equation}
The $f(\theta)$ is continuous on $[ -\pi, \pi)$, therefore, for arbitrary of $\theta$, this Fourier series converge to $f(\theta)$. \footnote{If the function is continuous, the series converges to the function; otherwise, it converges to the midpoint of the jump, i.e, the left and right hand limits.}
}
\solve{
(b) \quad $f(\theta)$ is neither even nor odd, the Fourier coefficients $a_{n},b_{n}$ can be written as
\begin{equation}
	\begin{aligned}
		a_{n} & = \frac{1}{\pi} \int_{-\pi}^{\pi} \cos ( n\theta ) f(\theta) d\theta   \\
		      & = \frac{1}{\pi} \int_{-\pi}^{\pi} \cos ( n\theta ) e^{\theta} d\theta,
	\end{aligned}
\end{equation}
\begin{equation}
	\begin{aligned}
		b_{n}
		 & = \frac{1}{\pi} \int_{-\pi}^{\pi} \sin (n\theta) f(\theta) d\theta   \\
		 & = \frac{1}{\pi} \int_{-\pi}^{\pi} \sin (n\theta) e^{\theta} d\theta.
	\end{aligned}
\end{equation}

\pf{$\dps \int e^{ax} \sin bx dx = \dfrac{e^{ax} ( a \sin bx - b \cos bx )}{a^{2} + b^{2}}$}{
	Let
	\begin{gather*}
		I = e^{ax} \sin bx,
	\end{gather*}
	and
	\begin{gather*}
		\begin{cases}
			u = \sin bx \\
			dv = e^{ax} dx
		\end{cases}
		\Rightarrow
		\begin{cases}
			du = b\cos bx \\
			v = \dfrac{e^{ax}}{a}
		\end{cases}
	\end{gather*}
	One can get
	\begin{equation*}
		\begin{aligned}
			\frac{e^{ax}}{a} \sin bx - \int \frac{e^{ax}}{a} b\cos bx dx.
		\end{aligned}
	\end{equation*}
	Let
	\begin{gather*}
		J  = \frac{e^{ax}}{a} b\cos bx dx,
	\end{gather*}
	and
	\begin{gather*}
		\begin{cases}
			u = b \cos b x \\
			dv = \dfrac{e^{ax}}{a} d x
		\end{cases}
		\Rightarrow
		\begin{cases}
			du = - b^{2} \sin bx \\
			v = \dfrac{e^{ax}}{a^{2}}
		\end{cases}
	\end{gather*}
	One can get
	\begin{equation*}
		\begin{aligned}
			J & = \frac{e^{ax}}{a^{2}}b \cos bx + \frac{b^{2}}{a^{2}} \int e^{ax} \sin bx dx \\
			  & = \frac{e^{ax}}{a^{2}}b \cos bx + \frac{b^{2}}{a^{2}} \int I dx ,            \\
		\end{aligned}
	\end{equation*}
	We can see this pattern is repeat. In particular,
	\begin{equation*}
		\begin{aligned}
			\int I dx                              & = \frac{e^{ax}}{a} \sin bx - \frac{e^{ax}}{a^{2}}b \cos bx - \frac{b^{2}}{a^{2}} \int I dx \\
			\Rightarrow ( a^{2} + b^{2} )\int I dx & = e^{ax} ( a \sin bx - b \cos bx ).
		\end{aligned}
	\end{equation*}
	This leads to
	\begin{equation*}
		\begin{aligned}
			\int e^{ax} \sin bx dx
			 & = \dfrac{e^{ax} ( a \sin bx - b \cos bx )}{a^{2} + b^{2}}. \\
		\end{aligned}
	\end{equation*}
}

\nt{
	\begin{equation*}
		\begin{aligned}
			\int e^{ax} \sin bx dx = \dfrac{e^{ax} ( a \sin bx - b \cos bx )}{a^{2} + b^{2}}, \\
			\int e^{ax} \cos bx dx = \dfrac{e^{ax} ( a \cos bx - b \sin bx )}{a^{2} + b^{2}}.
		\end{aligned}
	\end{equation*}
}

One can get
\begin{equation}
	\begin{aligned}
		a_{n}
		 & = \frac{1}{\pi} \evaluated{\frac{e^{\theta} ( \cos n\theta - n \sin n \theta )}{1 + n^{2}}}_{-\pi}^{\pi} \\
		 & = \frac{1}{\pi ( 1 + n^{2} )} \left( e^{\pi} \cos( n \pi ) - e^{-\pi} \cos( - n\pi ) \right)             \\
		 & = \frac{( -1 )^{n}}{\pi ( 1 + n^{2} )} \left( e^{\pi} - e^{-\pi} \right),
	\end{aligned}
\end{equation}
and
\begin{equation}
	\begin{aligned}
		b_{n}
		 & = \frac{1}{\pi} \evaluated{\frac{ e^{\theta} ( \sin n\theta - n \cos n \theta ) }{1 + n^{2}} }_{-\pi}^{\pi} \\
		 & = \frac{n}{\pi ( 1 + n^{2} )} \left( - e^{\pi} \cos( n \pi ) + e^{-\pi} \cos( - n\pi ) \right)              \\
		 & = -\frac{n( -1 )^{n}}{\pi ( 1 + n ^{2} )} \left( e^{\pi} - e^{-\pi} \right).
	\end{aligned}
\end{equation}
$a_{0}$ can be written as
\begin{equation}
	\begin{aligned}
		a_{0}
		 & = \frac{1}{\pi} \int_{-\pi}^{\pi} f(\theta) d\theta  \\
		 & = \frac{1}{\pi} \int_{-\pi}^{\pi} e^{\theta} d\theta \\
		 & = \frac{1}{\pi} ( e^{\pi} - e^{-\pi} ).
	\end{aligned}
\end{equation}
The Fourier series of $f(\theta) = e^{\theta}$ can be written as
\begin{equation}
	\label{eq: FS of e^theta}
	\begin{aligned}
		e^{\theta}
		 & \simeq \frac{1}{2 \pi} ( e^{\pi} - e^{-\pi} ) + \sum_{n = 1}^{\infty} \biggl[ \frac{( -1 )^{n}}{\pi ( 1 + n^{2} )} \left( e^{\pi} - e^{-\pi} \right) \cos n \theta \\
		 & \quad - \frac{n( -1 )^{n}}{\pi ( 1 + n ^{2} )} \left( e^{\pi} - e^{-\pi} \right) \sin n\theta \biggr],
	\end{aligned}
\end{equation}
where
\begin{gather*}
	a_{0} = \frac{1}{\pi} ( e^{\pi} - e^{-\pi}),                                                 \\
	a_{n} = \frac{( -1 )^{n}}{\pi ( 1 + n^{2} )} \left( e^{\pi} - e^{-\pi} \right) \cos n \theta, \\
	b_{n} = - \frac{n( -1 )^{n}}{\pi ( 1 + n ^{2} )} \left( e^{\pi} - e^{-\pi} \right) \sin n\theta.
\end{gather*}
At discontinuity, function $f(\theta) = e^{\theta}$ converges to the average of the left and right limits. In particular,
\begin{gather}
	S_{n} = \frac{f(\pi^{-}) + f(\pi^{+})}{2} = \frac{e^{\pi} + e^{-\pi}}{2} = \cosh \pi
\end{gather}
Therefore, one can rewritte Eq.~\eqref{eq: FS of e^theta} with $\theta = \pi$ as
\begin{equation}
	\begin{aligned}
		\cosh \pi 
		&= \frac{\sinh \pi}{\pi} + \frac{2\sinh \pi}{\pi} \sum_{n = 1}^{\infty} \frac{(-1)^{n}}{1+n^{2}} \cos n \pi - \frac{n(-1)^{n}}{(1+n^{2})} \sin \pi \\ 
		&= \frac{\sinh \pi}{\pi} + \frac{2\sinh \pi}{\pi} \sum_{n = 1}^{\infty} \frac{(-1)^{n}}{1+n^{2}} (-1)^{n}\\
		\Rightarrow & \sum_{n = 1}^{\infty} \frac{1}{1 + n^{2}} = \frac{1}{2}(\pi \coth \pi - 1).
	\end{aligned}
\end{equation}
}
\solve{
	(c) \quad $f(\theta) = \theta e^{i\theta}$ is neither even nor odd, one can obtain the Fourier series by
	\begin{gather}
		\theta e^{i\theta} = \frac{a_{0}}{2} + \sum_{n=1}^{\infty} a_{n} \cos n\theta + b_{n} \sin n \theta,
	\end{gather}
	where
	\begin{gather}
		a_{0} = \frac{1}{\pi} \int_{-\pi}^{\pi} \theta e^{i\theta} d\theta ,\\
		a_{n} = \frac{1}{\pi} \int_{-\pi}^{\pi} \theta \cos n \theta e^{i \theta} d\theta ,\\
		b_{n} = \frac{1}{\pi} \int_{-\pi}^{\pi} \theta \sin n \theta e^{i \theta} d\theta .
	\end{gather}
	\nt{
		\begin{equation*}
			\begin{aligned}
				\int x e^{ax} \sin bx dx = \dfrac{x e^{ax} ( a \sin bx - b \cos bx )}{a^{2} + b^{2}} - \frac{e^{ax} \left[ ( a^{2} - b^{2} )\sin bx - 2ab \cos bx \right]}{( a^{2} + b^{2} )^{2}}, \\
				\int x e^{ax} \cos bx dx = \dfrac{x e^{ax} ( a \cos bx - b \sin bx )}{a^{2} + b^{2}} - \frac{e^{ax} \left[ ( a^{2} - b^{2} )\cos bx + 2ab \sin bx \right]}{( a^{2} + b^{2} )^{2}}.
			\end{aligned}
		\end{equation*}
	}
	In particular,
	\begin{gather}
		a_{0} = -i e^{i\pi}( \pi + i ) + i e^{-i\pi}( -\pi + i ) = 2 i \pi ,
	\end{gather}
	and
	\begin{equation}
		\begin{aligned}
			a_{n} & = \frac{1}{\pi} \int_{-\pi}^{\pi}\theta \cos n \theta e^{i \theta} d\theta ,                                                                                                                                                          \\
			      & = \evaluated{\frac{\theta e^{i\theta}( i \cos n\theta - n \sin n \theta )}{n^{2} - 1}}_{-\pi}^{\pi} - \evaluated{\frac{e^{i\theta}\left[ ( - n^{2} - 1 )\cos n \theta + 2 i n \sin n \theta \right]}{( n^{2} - 1 )^{2}}}_{-\pi}^{\pi} \\
			      & = -\frac{2 \pi i ( -1 )^{n}}{n^{2} - 1}.
		\end{aligned}
	\end{equation}
	\begin{equation}
		\begin{aligned}
			b_{n} & = \frac{1}{\pi} \int_{-\pi}^{\pi}\theta \sin n \theta e^{i \theta} d\theta                                                                                                                                                              \\
			      & = \evaluated{ \frac{\theta e^{i\theta}( i \sin n\theta - n \cos n \theta )}{n^{2} - 1} }_{-\pi}^{\pi} - \evaluated{\frac{e^{i\theta}\left[ ( - n^{2} - 1 )\sin n \theta - 2 i n \cos n \theta \right]}{( n^{2} - 1 )^{2}}}_{-\pi}^{\pi} \\
			      & = \frac{2 \pi n ( -1 )^{n}}{n^{2} - 1}.
		\end{aligned}
	\end{equation}
	Therefore, one can obtain the Fourier series for $f(\theta) = \theta \cos \theta$ and $f(\theta) = \theta \sin \theta$ by
	\begin{gather}
		\theta e^{i\theta} = \theta \cos \theta  + i \theta \sin \theta  .
	\end{gather}
	where
	\begin{gather}
		\theta \cos \theta  = \Re[\frac{a_{0}}{2}] + \sum_{n=1}^{\infty} \Re[a_{n}]\cos n \theta + \Re[b_{n}] \sin n \theta,\\
		\theta \sin \theta = \Im[\frac{a_{0}}{2}] + \sum_{n=1}^{\infty} \Im[a_{n}]\cos n \theta + \Im[b_{n}] \sin n \theta.
	\end{gather}
	In particular
	\begin{gather}
		\theta \cos \theta = \sum_{n = 1}^{\infty} \frac{2 \pi n ( -1 )^{n}}{n^{2} - 1} \sin n \theta,
	\end{gather}
	and
	\begin{gather}
		\theta \sin \theta  = 2 \pi - \sum_{n = 1}^{\infty} \frac{2 \pi n ( -1 )^{n}}{n^{2} - 1} \cos n \theta.
	\end{gather}
}
\newpage
\begin{problem}{David Skinner, 1B Methods, Page 1}{p2}
A certain function $\vartheta( x,t )$ obeys the condition
\begin{equation*}
	\begin{aligned}
		\vartheta( x + 1, t )            & = \vartheta( x,t )                      \\
		\vartheta( x + it, t )            & = e^{\pi t - 2 \pi i x}\vartheta( x,t ) \\
		\int_{0}^{1} \vartheta( x,t ) dx & = 1.
	\end{aligned}
\end{equation*}
\begin{itemize}
	\item[a)] Using the first condition. Find the Fourier series with some unknown.
	\item[b)] Use the remaining conditions to fix these coefficients. For what range of $t$ does the series converge?
	\item[c)] Show that
	      \begin{gather*}
		      \frac{\partial \vartheta(x,t)}{\partial t} = \frac{1}{4\pi} \frac{\partial^{2} \vartheta(x,t)}{\partial x^{2}}.
	      \end{gather*}
\end{itemize}
\end{problem}

\solve{
	(a) \quad The Fourier series of the function $\vartheta(x,t)$ can be written as
	\begin{gather*}
		\vartheta(x) = \frac{a_{0}}{2} + \sum_{n=1}^{\infty} a_{n} \cos n x + b_{n} \sin n x,
	\end{gather*}
	where
	\begin{gather*}
		a_{0} = \int_{c}^{c + 1} \vartheta(x,t) dx,
	\end{gather*}
	\begin{gather*}
		a_{n} = \int_{c}^{c + 1} \vartheta(x,t) \cos (n \pi x)  dx,
	\end{gather*}
	\begin{gather*}
		b_{n} = \int_{c}^{c + 1} \vartheta(x,t) \sin (n \pi x)  dx.
	\end{gather*}
}
\solve{
	(b) \quad Using the remaining conditions, one can get
	\begin{equation}
		\begin{aligned}
			a_{0}
			&= \int_{c}^{c + 1} \vartheta(x,t) dx \\
			&= \int_{c}^{c+1} \vartheta(x,t) dx \\
			&= \int_{0}^{1} \vartheta(x,t) dx\\
			&= 1.
		\end{aligned}
	\end{equation}
	In particular, one can obtain the complex Fourier series of $\vartheta(x,t)$ by
	\begin{gather}
		\vartheta(x,t) = \sum_{n = -\infty}^{\infty} c_{n}(t) e^{i 2 \pi n x},
	\end{gather}
	using the normalization condition, we get 
	\begin{equation}
		\begin{aligned}
			\int_{0}^{1} \vartheta(x,t) dx 
			&= \int_{0}^{1} \sum_{n = -\infty}^{\infty} c_{n}(t) e^{i 2 \pi n x} dx \\
			& = \sum_{n = - \infty}^{\infty} \int_{0}^{1} c_{n}(t) e^{2 \pi i n x} dx, \\
		\end{aligned}
	\end{equation}
	with $n = 0$, we can easily get $c_{0} = 1$ and with $n \neq 0$, the integral $\int_{0}^{1} e^{2 \pi i n x} dx = 0$. In particular
	\begin{gather}
		\vartheta(x,t) = 1 + \sum_{\substack{n = -\infty \\ n \neq 0}}^{\infty} c_{n}(t) e^{2 \pi i n x} dx
	\end{gather}
	In addition, with the second condition, we can get 
	\begin{equation}
		\begin{aligned}
			\vartheta(x + it, t) 
			&= e^{\pi t - 2 \pi i x} \vartheta(x ,t) \\ 
			&= e^{\pi t - 2 \pi i x} \sum_{n = - \infty}^{\infty} c_{n}(t) e^{2 \pi i n x} \\ 
			& = \sum_{n = - \infty}^{\infty} c_{n}(t) e^{2 \pi i (n - 1) x} e^{\pi t} \\
			& = \sum_{n = \infty}^{\infty} c_{n + 1}(t) e^{2 \pi i n} e^{\pi t}
		\end{aligned}
	\end{equation}
	however
	\begin{equation}
		\begin{aligned}
			\vartheta(x + it, t) 
			& = \sum_{n = - \infty}^{\infty} c_{n}(t) e^{2 \pi i n (x + it)} \\
			& = \sum_{n = - \infty}^{\infty} c_{n}(t) e^{2 \pi i n x} e^{- 2 \pi n t},\\
		\end{aligned}
	\end{equation}
	one can easily obtain the recursive formula of $c_{n}$ by
	\begin{gather}
		\label{eq:recursion of cn}
		c_{n + 1} = c_{n} e^{- \pi t (2 n + 1)}.
	\end{gather}
	In particular,
	\begin{equation}
		\begin{aligned}
			c_{1} & = c_{0} e^{-\pi t},\\
			c_{2} & = c_{1} e^{-\pi t(2 + 1)} = c_{0} e^{-4\pi t},\\
			c_{3} & = c_{2} e^{-\pi t(4 + 1)} = c_{0} e^{-9\pi t},\\
			c_{4} & = c_{3} e^{-\pi t(6 + 1)} = c_{0} e^{-16\pi t},\\
			\vdots & \\
			c_{k} & = c_{0} e^{- k^{2} \pi t }, \\
			c_{k+1} & = c_{k} e^{-\pi t (2 k + 1)} = c_{0} e^{- k^{2} \pi t} e^{-\pi t (2 k + 1)} = c_{0} e^{-(k+1)^{2} \pi t}. \\
		\end{aligned}
	\end{equation}
	The \eqref{eq:recursion of cn} can be written as
	\begin{gather}
		c_{n} = c_{0} e^{- n^{2} \pi t},
	\end{gather}
	where $c_{0} = 1$. The Fourier series of $\vartheta(x,t)$ can be written as
	\begin{equation}
		\begin{aligned}
			\vartheta(x,t) 
			&= \sum_{n = -\infty}^{\infty} c_{n}(t) e^{2 \pi i n x} \\
			& = c_{0} + \sum_{n = -1}^{-\infty} c_{n} e^{2 \pi i nx} + \sum_{n = 1}^{\infty} c_{n} e^{2 \pi i n x} \\
			& = 1 + \sum_{n = -1}^{-\infty} c_{n} e^{2 \pi i nx} + \sum_{n = 1}^{\infty} c_{n} e^{2 \pi i n x} \\
			& = 1 + \sum_{n = 1}^{\infty} e^{-n^{2} \pi t} (e^{2 \pi i n x} + e^{-2 \pi i n x}) \\
			& = 1 + 2 \sum_{n = 1}^{\infty} e^{-n^{2} \pi t} \cos 2 \pi n x.\\
		\end{aligned}
	\end{equation}
	We then investigate whether this sum is converges. Let me consider $t = 0$, this series becomes to
	\begin{gather*}
		\vartheta(x,t) = 1 + 2 \sum_{n = 1}^{\infty} \cos 2 \pi n x,
	\end{gather*}
	The series diverges for $t =0$ because
	\begin{gather*}
		\lim\limits_{n \to \infty} \cos 2 \pi n x \neq 0 \quad \text{for} \; x \in \mathbb{Z}.
	\end{gather*}
	For $t > 0$, the exponential factor $e^{- n^{2} \pi t}$ tends to zero much faster the cosine term, which ensure convergence of the series. In constrast, when $t<0$, the exponetial factor grows exponentially, causing the series to diverge.
}

\begin{problem}{Skinner, 1B Methods, Page 1,2}{p3}
The \textit{sawtooth function} is defined to be the function
\begin{gather*}
	f(\theta) = \theta  ,
\end{gather*}
for $\theta \in [ -\pi,\pi )$.
\begin{itemize}
	\item[a)] Compute the Fourier series of the sawtooth function and comment on its value at $\theta = \pi$.
	\item[b)] By applying Parseval's identity to the sawtooth function, show that
	      \begin{gather*}
		      \frac{\pi^{2}}{6} = \sum_{n=1}^{\infty} \frac{1}{n^{2}}
	      \end{gather*}
\end{itemize}
\end{problem}

\solve{
(a) \quad The Fourier series of the sawtooth function can be written as
\begin{gather}
	f(\theta) = \theta = \sum_{n=1}^{\infty} a_{n} \cos n\theta + b_{n} \sin n \theta,
\end{gather}
where
\begin{equation}
	\begin{aligned}
		a_{0} &= 0 , \\ 
	\end{aligned}
\end{equation}
\begin{gather}
	a_{n} = \frac{1}{\pi}\int_{-\pi}^{\pi} \theta \cos (n \theta) d\theta = 0,
\end{gather}
\begin{equation}
	\begin{aligned}
		b_{n} & = \frac{1}{\pi} \int_{-\pi}^{\pi} \theta \sin (n \theta) d\theta \\
		& = \frac{2}{\pi} \int_{0}^{\pi} \theta \sin (n \theta) d\theta \\ 
		& = \frac{2}{\pi} \left( \eval{- \theta \frac{\cos (n \theta)}{n}}_{0}^{\pi} + \int_{0}^{\pi} \frac{\cos (n \theta)}{n} d\theta \right) \\
		& = - \frac{2}{n} \cos n \pi = \frac{2(-1)^{n+1}}{n}.
	\end{aligned}
\end{equation}
}
\solve{
(b) \quad One can write the Fourier series of sawtooth function as
\begin{gather}
	f(\theta) = \sum_{n = 1}^{\infty} \frac{2(-1)^{n+1}}{n} = \sum_{-\infty}^{\infty} \frac{i(-1)^{n}}{n}.
\end{gather}
Because $f(\theta) = f(\theta)^{*}$, by applying the Parseval's identity to the sawtooth function, one can get
\begin{gather}
	\label{eq:lhs parseval}
	\int_{-\pi}^{\pi} \theta^{2} d\theta = \frac{2 \pi^{3}}{3},
\end{gather}
while
\begin{gather}
	\label{eq:rhs parseval}
	2 \pi \sum_{n \in \mathbb{Z}} |b_{n}|^{2} = 4 \pi \sum_{n = 1}^{\infty} \frac{1}{n^{2}}.
\end{gather}
Using Eq.~\eqref{eq:lhs parseval} and Eq.~\eqref{eq:rhs parseval}, one can get
\begin{gather}
	\frac{\pi^{2}}{6} = \sum_{n = 1}^{\infty} \frac{1}{n^{2}}.
\end{gather}
}
\begin{problem}{David Skinner, 1B Methods, page 2}{p4}
	The \textit{square wave} is defined by
	\begin{gather}
		f(\theta) =
		\begin{cases}
			1 \quad \text{for}\; \theta \in (0,\pi), \\
			0 \quad \text{for}\; \theta \in (-\pi,0), \\
		\end{cases}
	\end{gather}
\end{problem}



\end{document}
