\documentclass[a4paper, twoside, twocolumn]{article}
%\usepackage[utf8]{vietnam}
\usepackage[utf8]{inputenc}
\usepackage{anyfontsize,fontsize}
\changefontsize[13pt]{13pt}
\usepackage{commath}
\usepackage[d]{esvect}
\usepackage{parskip}
\usepackage{xcolor}
\usepackage{amssymb}
\usepackage{slashed,cancel}
\usepackage{indentfirst}
\usepackage{pdfpages}
\usepackage{graphicx}
\usepackage{upgreek}
\usepackage{nccmath,nicematrix}
\usepackage{mathtools}
\usepackage{amsmath,systeme,amsthm,amsfonts}
\usepackage[thinc]{esdiff}
\usepackage{hyperref}
\usepackage{bm,physics}
\usepackage{fancyhdr}
\usepackage{setspace}
\usepackage{lipsum}

\usepackage[most,many,breakable]{tcolorbox}

\definecolor{mytheorembg}{HTML}{F2F2F9}
\definecolor{mytheoremfr}{HTML}{00007B}
%footnote
\pagestyle{fancy}
\fancyhf{}
\renewcommand{\headrulewidth}{0pt}
\fancyhfoffset{0pt}
\fancyfoot[LE]{Theoretical Physics}
\fancyfoot[RO]{A note on Mathematical Methods}
\fancyfoot[RE,LO]{\thepage}


\usepackage{geometry}
\geometry{top=1cm,bottom=1.5cm,left=2cm,right=2cm,includehead,includefoot}
\setlength{\columnsep}{7mm}

\newcommand{\image}[1]{
\begin{figure}[H]
  \centering
  \includegraphics[width=8.0cm,height=5.0cm]{pic/#1}
  \label{#1}
\end{figure}
}
\renewcommand{\l}{\ell}
\newcommand{\dps}{\displaystyle}
\newcommand{\mean}[1]{\langle{#1}\rangle}
\newcommand{\f}[2]{\dfrac{#1}{#2}}
\newcommand{\at}[2]{\bigg\rvert_{#1}^{#2} }

\newcommand{\Qed}{\begin{flushright}\qed\end{flushright}}

\newcommand{\solve}[1]{\setlength{\parindent}{0cm}\textbf{\textit{Solution: }}\setlength{\parindent}{1cm}#1 \Qed}

\renewcommand{\baselinestretch}{2.0}

\tcbuselibrary{theorems,skins,hooks}
\newtcbtheorem{problem}{Problem}
{%
  enhanced,
  breakable,
  colback = mytheorembg,
  frame hidden,
  boxrule = 0sp,
  borderline west = {2pt}{0pt}{mytheoremfr},
  sharp corners,
  detach title,
  % before upper = \tcbtitle\par\smallskip,
  before upper* = \tcbtitle \par\smallskip,
  coltitle = mytheoremfr,
  fonttitle = \bfseries\sffamily,
  description font = \mdseries,
  separator sign none,
  segmentation style={solid, mytheoremfr},
}
{p}

\title{\Huge{A note on Mathematical Methods}}

\hypersetup{
  colorlinks=true,
  linkcolor=blue,
  filecolor=magenta,
  urlcolor=blue!70!red,
  pdftitle={MM},
  pdfpagemode=FullScreen,
}

\urlstyle{same}

\setstretch{1.0}
\renewcommand{\abstractname}{}
\begin{document}
\setlength{\parindent}{20pt}
\newpage
\author{Tran Khoi Nguyen \\Department of Theoretical Physics}
\twocolumn[
\maketitle
\begin{abstract}
	These notes are a review of the mathematical methods course, focusing on the content most relevant for physics. The primary sources were mostly come from David Skinner's \href{https://www.damtp.cam.ac.uk/user/dbs26/1Bmethods.html}{lecture notes on Methods}.
\end{abstract}
\newline
]
\thispagestyle{fancy}
\section{Fourier series}
\noindent We begin by reviewing Fourier series. Fourier series are defined for functions $f : S^{1} \to \mathbb{C}$ , parametrized by $\theta \in [-\pi, \pi)$. We defined the Fourier coefficients by an inner product
	\begin{gather*}
		\hat{f}_{n} = \frac{1}{2 \pi} \left(e^{in\theta}, f\right) \equiv \frac{1}{2 \pi} \int_{0}^{2\pi} e^{-i n \theta} f(\theta) d\theta.
	\end{gather*}
We then claim that
\begin{gather*}
	f(\theta) = \sum_{n \in \mathbb{Z}} \hat{f}_{n} e^{i n \theta}.
\end{gather*}
In particular, via reality condition, the Fourier series can be obtained by
\begin{gather*}
	f(\theta) = \frac{a_{0}}{2} + \sum_{n=1}^{\infty} a_{n} \cos n\theta + b_{n} \sin n \theta,
\end{gather*}
where
\begin{gather*}
	a_{0} = \frac{1}{\pi} \int_{-\pi}^{\pi} f(\theta) d\theta ,\\
	a_{n} = \frac{1}{\pi} \int_{-\pi}^{\pi} \cos n \theta f(\theta) d\theta ,\\
	b_{n} = \frac{1}{\pi} \int_{-\pi}^{\pi} \sin n \theta f(\theta) d\theta .
\end{gather*}
We then investigate whether this sum converges to $f$, if it converges at all. One can show that the Fourier series converges to $f$ for continuos functions with bounded continuos derivaties. When $f$ has a discontinuity, the Fourier series converges to the average of the left and right limits.
\subsubsection*{Fejer's theorem}
\noindent Fejer's theorem state that one can always recover $f$ from the $\hat{f}_{n}$ as long as $f$ is continuos except at finitely many points, though it makes no statement about the convergence of the Fourier series. Also, the Fourier series converges to $f$ as long as $\sum_{n} |\hat{f}_{n}|$ converges. In other words,
\begin{gather*}
	\lim\limits_{n \to \infty} a_{n} = 0, \\
	\lim\limits_{n \to \infty} b_{n} = 0.
\end{gather*}
\subsubsection*{Sawtooth function}
\noindent The sawtooth function defined by 
\begin{gather*}
	f(\theta) = \theta \quad for \quad \theta \in [-\pi, \pi),
\end{gather*}
and the Fourier coefficients for it are
\begin{gather*}
	a_{0} =  0, \quad n = 0,\\
	a_{n} = \frac{1}{in}(-1)^{n+1},\quad n \neq 0.
\end{gather*}
\subsubsection*{Parseval's identity}
\begin{gather*}
	\int_{-\pi}^{\pi} \abs{f(\theta)}^{2} d\theta = 2 \pi \sum_{k \in \mathbb{Z}} |\hat{f}_{k}|^{2}.
\end{gather*}
\lipsum[20]
The Fourier transform is linear, and obeys
\begin{gather*}
	\mathcal{F}[ f(x-a) ] = e^{-i k a} \tilde{f}( k ), \\
	\mathcal{F}[ e^{ilx} f(x) ] =  \tilde{f}( x ), \\
	\mathcal{F}[ f(cx) ] = \frac{\tilde{f}( k/c )}{\abs{c}}
\end{gather*}



\end{document}
